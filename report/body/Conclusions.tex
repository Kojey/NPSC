\chapter{Conclusions and Recommendations}

\section{Review of objectives}
The sub-requirements set in \ref{sub_requirements} and their satisfaction level are shown in \cref{table:conclusion_subrequirements}. The table provides reasons on the satisfaction level given for each requirement.
\begin{table}[h!]
	\centering
	\caption{Sub-requirements implementation satisfaction.}
	\label{table:conclusion_subrequirements}
	\begin{tabular}{p{10em}cp{15em}}
		\hline
		\hline
		\toprule
		\textbf{Sub-requirements} & \textbf{Level of satisfaction} & \textbf{Coments}\\
		\bottomrule
		\toprule
		Onboard touchscreen & High & Good framework design for communication between screen and STM\\
		\midrule
		Smartphone App & Low & Framework is developed but actual application can only connect to bluetooth\\
		\midrule
		User preference & Medium & Framework for storing and loading any parameters have been implemented\\
		\midrule
		Set/edit alarm & High& Good framework allowing alarm configuration\\
		Set/get time and date & High & Good framework allowing time and date parameters to be stored and loaded\\
		\midrule
		Light parameters & Good & Framework allows configuration of the neopixels colour and brightness.\\
		\midrule
		Light pattern & Low & Data sent to neopixels is corrupted by the interrupt generated by other module\\
		\bottomrule
		\hline
		\hline
	\end{tabular}
\end{table}
Each of these sub-requirements is related to the system requirement. \Cref{table:conclusion_requirements} provides the satisfaction lvel of the system requirements. 
\begin{table}[h!]
	\centering
	\caption{Requirements implementation satisfaction.}
	\label{table:conclusion_requirements}
	\begin{tabular}{cc}
		\hline
		\hline
		\toprule
		\textbf{Requirements} & \textbf{Level of satisfaction}\\
		\bottomrule
		\toprule
		Visual & Low \\
		\midrule
		Instruction & High \\
		\midrule
		Alarm & High \\
		\bottomrule
		\hline
		\hline
	\end{tabular}
\end{table}
\subsection{Instruction and user inputs}
The instruction application provided a solid foundation for the implementation of the applications running on each input device. The Nextion touchscreen application provides a nice and easy user interaction with the NPSC. The smartphone application is not fully implemented, the basic interaction provided by this application was connecting the smartphone to the NPSC system. 
\subsection{Alarm}
The alarm application successfully managed the time, date and alarm information of the system. The user has full control of the time, date and the alarm to be set.  
\subsection{Visual}
The visual module was able to control the neopixel at a module level. However, it failed once it was integrated with other modules as they interupted the stream of data sent to the neopixel. For this reason the visual application could not be fully implemented.

\paragraph{Objectives}
\textit{The purpose of this study is to create a device that can be used to regulate the human sleep-wake cycle while being user-friendly and a personalisable digital alarm clock.} \\
The objectives are reviewed in \Cref{table:conclusion_objectives}.
\begin{table}[h!]
	\centering
	\caption{Review of objectives.}
	\label{table:conclusion_objectives}
	\begin{tabular}{p{20em}c}
		\hline
		\hline
		\toprule
		\textbf{Objectives} & \textbf{Comment}\\
		\bottomrule
		\toprule
		The device is capable of producing light of $460\pm10nm$ wavelength at an illuminance of $30lx$ & \checkmark \\
		\midrule
		The device is an alarm clock & \checkmark \\
		\midrule
		The device is user-friendly & \checkmark \\
		\midrule
		The device has more features than its competitors & X \\
		\bottomrule
		\hline
		\hline
	\end{tabular}
\end{table}  

\section{Reflections on the design}
The prototype of the NPSC is proof that the neopixels can be used to create a device capable of affecting the human sleep-wake cycle. However, using a large number of neopixels in an embedded system requires careful use of the microcontroller resources. In this project, the DMA was not used to control the neopixel resulting in the CPU doing all the work from the communication between all modules to the transmission of the neopixels' data. \\
The prototype is subjective to many design changes, therefore, a cost analysis could not be performed. Considering all the module designed for the neopixel, a significant amount of work has been done. All modules have been tested and proven to work creating a solid foundation for future improvements. 

\section{Recommendation for future work}
The following recommendations are firstly made so that all requirements established in the introduction are met, secondly do that the NPSC's design and performance increases over time.\\
The recommendations are the following:
\begin{itemize}
	\item The NPSC's visual module should be dedicated to another microcontroller, preferably an STM32F0 so that the libraries made can be reused. The role of the STM32F0 would solely be to read instruction from the STM32F4 and update the visual output accordingly. This would reduce the workload of the STM32F4 and physically separate the visual module from the other module. 
	\item The neopixel library should make use of the Direct Memory Access (DMA) to reduce the CPU workload.
	\item The experiment performed on the Ring should be done in a completely dark room where the reflection would be minimum to increase the reading accuracy
\end{itemize}