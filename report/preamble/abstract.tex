{\Large Abstract}\\
\hrule
\vskip 10mm
Studies made on the human behavioural patterns revealed the existence of an endogenous clocks controlling the human circadian rhythm. Among these clocks, the master clock located in the suprachiasmatic nuclei responsible for the control of the human sleep-wake cycle is influenced by 
light. \\
This project aims to create a device capable of emitting light which having both soporific and alarming effects on human beings in order to control the human sleep-wake cycle. The device is expected to be an alarm clock which can be used as a supplement to the cure of sleep disorder.\\
The design has three focus areas: the hardware, software and mechanical design. The hardware design consists of the identification and selection of components required to make the device an alarm clock that can meet the light requirements. The mechanical design is paired with the hardware design to ensure that the device can fit in a $25*25*10cm^3$ case. The software design consists of identifying the abstract modules and designing when required communication protocols between these modules. in this section, it was decided that the device will use a ring consisting of 180 neopixels as its light source, will require external storage capacity and a real time clock with an alarm functionality, will be controlled by an onboard touchcreen and a smartphone application.\\
The system was tested using software unit tests and a logic analyser for the hardware modules. The light source illuminance, current drawn per colour and temperature rise were quantified during experiments.\\
The ring of neopixels performed well above expectations; it emits light of $465-467nm$ wavelength with an illuminance above 30lx on any object placed at maximum of 1m at a maximum angle of 90 degree to the normal of the ring surface. 

 