Use the IEEE numbered reference style for referencing your work as shown in your thesis guidelines.
Please remember that the majority of your referenced work should be from journal articles, technical
reports and books not online sources such as Wikipedia.

\begin{thebibliography}{5}
% GE Lighting, lighting and sleep
\bibitem{ge2014} General Electric Company, ``GE Lighting, lighting and sleep'', December 2014.
% Circadian and Light Effects on Human Sleepiness-Alertness
\bibitem{cir2014} C. Cajochen, S. L. Chellappa and C. Schmidt, ``Circadian and Light Effects on Human Sleepiness-Alertness'', \emph{Sleepiness and Human Impact Assessment}, pp. 9-22, 2014.
% Light, Melatonin and Sleep-Wake Cycle
\bibitem{lig1994} Gregory M. Brown, ``Light, Melatonin and Sleep-Wake Cycle'', \emph{Journal Psychriatry Neurosci}, {\bf vol. 19(5)}, pp. 345-353, Nov 1994.
% Blue light from light-emitting diodes elicits a dose-dependent suppression of melatonin in humans
\bibitem{bl2010} K. E. West, M. R. Jablonski, B. Warfield, K. S. Cecil, M. James, M. A. Ayers, J. Maida, C. Bowen, D. H. Sliney, M. D. Rollag, J. P. Hanifn and G. C. Brainard, ``Blue light from light-emitting diodes elicits a dose-dependent suppression of melatonin in humans'', \emph{Journal Appl Physiol}, {\bf vol. 110}, pp. 619-626, 16 Dec 2010.
% Action Spectrum for Melatonin Regulation in Humans: Evidence for a Novel Circadian Photoreceptor
\bibitem{ac2001} George C. Brainard, John P. Hanifin, Jeffrey M. Greeson, Brenda Byrne, Gena Glickman, Edward Gerner and Mark D. Rollag, ``Action Spectrum for Melatonin Regulation in Humans: Evidence for a Novel Circadian Photoreceptor'', \emph{Journal of Neuroscience}, {\bf vol. 21(16)}, pp. 6405-6412, 15 Au 2001.
% Phototransduction by Retinal Ganglion Cells That Set the Circadian Clock.
\bibitem{ph2002} Berson, D. M., F. A. Dunn, and M. Takao. ``Phototransduction by Retinal Ganglion Cells That Set the Circadian Clock.'' \emph{Science}, {\bf vol. 295}, pp. 1070-073, 2002.
% An action spectrum for melatonin suppression: evidence for a novel non-rod, non-cone photoreceptor system in humans
\bibitem{an2001} Kavita Thapan, Josephine Arendt and Debra J. Skene, ``An action spectrum for melatonin suppression: evidence for a novel non-rod, non-cone photoreceptor system in humans'', \emph{Journal of Physiology}, {\bf vol. 535.1}, pp.261-267, July 2001.
% Dose-response  relationship  for  light  intensity  and  ocular  and electroencephalographic  correlates  of  human  alertness
\bibitem{do2000} Christian  Cajochen*,  Jamie  M.  Zeitzer,  Charles  A.  Czeisler,  Derk-Jan  Dijk, ``Dose-response  relationship  for  light  intensity  and  ocular  and electroencephalographic  correlates  of  human  alertness'', \emph{Behavioural Brain Research }, {\bf vol. 115}, pp.75-83, May 2000.
% Early versus late bedtimes phase shift the human dim light melatonin rhythm despite a fixed morning lights on time
\bibitem{ea2004} Helen J. Burgess and Charmane I. Eastman, ``Early versus late bedtimes phase shift the human dim light melatonin rhythm despite a fixed morning lights on time'', \emph{Neurosci Lett.}, {\bf vol. 356(2)}, pp. 115–118, Feb 2004.
% Home Lighting Before Usual Bedtime Impacts Circadian Timing: A Field Study
\bibitem{ho2014} Helen J. Burgess and Thomas A. Molina, ``Home Lighting Before Usual Bedtime Impacts Circadian Timing: A Field Study'', \emph{Photochem Photobiol.}, {\bf vol. 90(3)}, pp. 723–726, 2014.
% Lighting for Health: LEDs in the New Age of Illumination
\bibitem{ho2014} US Department of Energy, ``Lighting for Health: LEDs in the New Age of Illumination'', \emph{Solid-State Lighting Technology Fact Sheet }, 2014.
% A Working Threshold for Acute Nocturnal Melatonin Suppression from “White”Light Sources used in Architectural Applications
\bibitem{aw2013} Mark S. Rea and Mariana G. Figueiro, ``A Working Threshold for Acute Nocturnal Melatonin Suppression from “White”Light Sources used in Architectural Applications'', \emph{Lighting Research Center, Rensselaer Polytechnic Institute, Troy, New York, USA}, 2013.
% Internal rhythms in humans
\bibitem{in1996} Derk-Jan Dijk, ``Internal rhythms in humans'', \emph{CELL \& DEVELOPMENTAL BIOLOGY}, {\bf vol. 7}, pp. 831-836, 1996.
% Scientific Background Discoveries of Molecular Mechanisms Controlling the Circadian Rhythm
\bibitem{sc2017} ``Scientific Background Discoveries of Molecular Mechanisms Controlling the Circadian Rhythm'', \emph{The Nobel Assembly at Karolinska Institutet}, 2017.
% Is sleep fundamentally different between mammalian species?
\bibitem{is1995} Tobler I., ``Is sleep fundamentally different between mammalian species?'', \emph{Behav Brain Res.}, {\bf vol. 69(1-2)}, pp. 35-41, Jul-Aug 1995.
% GE Sol
\bibitem{gesol} GE sol, [Online], available at: https://www.cbyge.com/products/sol
% Philips
\bibitem{philips} Philips, [Online], available at: https://www.usa.philips.com/c-p/HF3550\_60/discontinued-wake-up-light/overview
\end{thebibliography}
